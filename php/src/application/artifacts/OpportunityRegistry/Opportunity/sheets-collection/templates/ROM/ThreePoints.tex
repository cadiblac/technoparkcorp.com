%%
%% Copyright TechnoPark Corp., 2010
%% @version $Id$
%%

<?=$this->estimate->estimator?> used \textbf{three-point estimate}
method, which is a simplified variant of
\href{http://en.wikipedia.org/wiki/Program_Evaluation_and_Review_Technique}{PERT}, 
where each feature is estimated with three points. Best Case ($BC$) is the amount
of staff-hours required to implement the feature, assuming everything proceeds better than is normally expected.
Worst Case ($WC$) is a pessimistic scenario, and Most Likely ($ML$) is the most
possible scenario. Cost is calculated with the formula:

\begin{equation}
C = \frac{BC + WC + 4 \times ML}{6} \times {<?=sprintf('%0.2f', $this->estimate->multiplier)?>}
\end{equation}

The multiplier \texttt{<?=sprintf('%0.2f', $this->estimate->multiplier)?>} indicates
our typical assumption that we will spend
approximately <?=round(100/$this->estimate->multiplier)?>\% of the budget for programming.
The numbers are:

\begin{equation}
C = \frac{<?=$this->estimate->bc?> + 
    <?=$this->estimate->wc?> + 
    4 \times <?=$this->estimate->ml?>}{6} \times {<?=$this->estimate->multiplier?>} =
    <?=$this->estimate->hours?>
\end{equation}
