%%
%% Copyright TechnoPark Corp., 2010
%% @version $Id$
%%

\section*{Quality Policy Basics}

TechnoPark Corp. is a proud 
<?=$this->href('about/news/year2008/iso9001', 'ISO-9001 certified company')?> and we
do all our best to fight for <?=$this->href('process/quality', 'quality')?> in our projects.

There are three dimensions of quality:
Quality of Process (referred to as ``QA'');
Quality of Project (calculated by ``SPI'');
Quality of Product (amount of discovered software defects).
Quality of product is the only dimension that we guarantee
to the customer according to the contract. The other two dimensions
are internal.

\textbf{Quality of process} is an average sum of 
270 quality metrics calculated
in semi-automatic mode once a week by QA Group. Each quality metric has
its weight (in 1-10 range). Each metric is rated from one to five, where
one is the lowest mark that means absolute non-compliance to the quality
requirement/metric.

The PM and the whole project team are responsible for raising the quality 
of process and keeping it high during the whole project lifecycle.
Our quality metrics are based on CMMI Dev v1.2. This approach is our 
patent-pending invention.

\textbf{Quality of project} includes business criteria like: 
on-time delivery, effectiveness of communication and in-budget completeness.

We use the \href{http://en.wikipedia.org/wiki/Earned_value_management}{Earned Value Analysis} 
technique to calculate the quality of project.
Schedule Performance Indicator (SPI) as a relation of Earned Value (EV) to Planned
Value (PV) is used for the indication of quality of project. The PM and the
whole team are responsible for keeping SPI higher than 4.2.

\textbf{Quality of product} is the degree of 
conformance of Deliverables to Specification.
Our goal in the project is to reveal as many defects as we can within
the project budget and fix critical ones. We believe that any software
has an unlimited amount of defects. All we can do is to reveal and fix the most critical
of them.

For quality definition we use 
\href{http://standards.ieee.org/reading/ieee/std_public/description/se/730-1998_desc.html}{IEEE 730}, 
\href{http://en.wikipedia.org/wiki/IEEE_829}{IEEE 829}, 
\href{http://standards.ieee.org/reading/ieee/std_public/description/se/1012-1998_desc.html}{IEEE 1012} 
and definitions from 
\href{http://standards.ieee.org/reading/ieee/std_public/description/se/610.12-1990_desc.html}{IEEE 610.12}.
QA Director is responsible for quality control in entire company,
<?=$this->href('mailto:qa@technoparkcorp.com', 'Maria Vinogradova')?>.
