\section*{Technical Capacity}

To perform a project we utilize several types of resources: project manager,
technicians, project management software, requirements specifier, users,
\href{http://en.wikipedia.org/wiki/Software_configuration_management}{SCM} system, 
knowledge, equipment and cash. Each resource is critical:

\begin{center}\begin{tikzpicture}
\tikzstyle{resource} = [font=\small, text width=25mm, text centered, anchor=center]
\tikzstyle{lbl} = [font=\scriptsize, text=tpcGreen]
\tikzstyle{ln} = [draw=tpcBlack, -triangle 60]

\node [draw=tpcBlue, fill=tpcBlue!30, outer sep=2mm] (project) {Project};
\node [shape=ellipse, minimum height=4cm, minimum width=8cm] at (project) (circle) {};

\node [below=0mm of circle.0, resource] (pm) {Project Manager};
\node [below=0mm of circle.40, resource] (soft) {Management Software};
\node [below=0mm of circle.90, resource] (scm) {SCM};
\node [below=0mm of circle.140, resource] (equipment) {Equipment};
\node [below=0mm of circle.180, resource] (technicians) {Technicians};
\node [below=0mm of circle.210, resource] (users) {Users};
\node [below=0mm of circle.260, resource] (specifier) {Requirements Specifier};
\node [below=0mm of circle.330, resource] (cash) {Cash};

\draw [ln] (pm) -- (project);
\draw [ln] (soft) -- (project);
\draw [ln] (scm) -- (project);
\draw [ln] (equipment) -- (project);
\draw [ln] (technicians) -- (project);
\draw [ln] (users) -- (project);
\draw [ln] (specifier) -- (project);
\draw [ln] (cash) -- (project);

\node [dashed, draw=tpcGreen, label={[lbl]-45:provided by you}, inner sep=0mm, thin, fit=(users) (specifier) (cash)] {};
\node [dashed, draw=tpcGreen, label={[lbl]-45:outsourced}, inner sep=0mm, thin, fit=(technicians)] {};
\end{tikzpicture}\end{center}

\textbf{Project Manager} is a trained individual working in TechnoPark Corp. on a
full-time basis. He/she possesses IBM Rational Unified Process (\href{http://en.wikipedia.org/wiki/IBM_Rational_Unified_Process}{RUP}) Certified Specialist
certificate, and Project Management Professional (\href{http://en.wikipedia.org/wiki/Project_Management_Professional}{PMP}) certificate. There are
five project managers in our company.

\textbf{Management Software} is a web toolkit used to create, maintain, and use project management artifacts
like Schedule, Cost, SRS, Risk List, Traceability Matrix, etc. We use our own management software
that is seamlessly integrated with Subversion and \href{http://continuum.apache.org/}{Apache Continuum}.
Our software is stable and mature (in development since December 2006, pending two USPTO patents).

\textbf{Software Configuration Management (SCM) software} is a system that keeps source code 
under versioning control. We use \href{http://subversion.tigris.org/}{Subversion (SVN)} -- one of a leading open source systems. Our SVN
repositories are located on dedicated servers provided by \href{http://www.webfaction.com}{WebFaction.com} -- one of the best
providers of such services. Regular (every 4 hours) backup procedure synchronizes SVN repositories with
their reserve copies stored in \href{http://aws.amazon.com/s3/}{Amazon S3}.

\textbf{Equipment} is rented from reliable providers, like \href{http://www.rackspace.com}{RackSpace.com} and WebFaction.com.
It is our rule not to possess any equipment, since we perform projects of different types and various technologies.
Equipment required for a particular project (like test deployment platform) usualy is rented for the
purpose of the project.

\textbf{Technicians} like programmers, designers, architects, testers, graphic artists and system administrators
are invited to the project from our supplying companies. There are over 30 companies in our
stock with various technologies, profiles and experience. 
Totally over 200 certified and experienced technicians are available for involvement in new projects.
We select the best suitable specialists
for each project, analyzing the experience, certificates and ability to be helpful in the project.

\textbf{Users} are people who will use the system after the project is finished. Users are provided
by you (at least user champions) and kept engaged during the whole project.

\textbf{Requirements Specifier} is a person who provides functional and non-functional requirements
for the project, approves SRS document and signs a project close-out report when the system is ready.
This person is a must in any project and is provided by you.
