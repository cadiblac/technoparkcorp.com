%%
%% Copyright TechnoPark Corp., 2010
%% @version $Id$
%%

<?=$this->estimate->estimator?> used \textbf{function-point analysis}
method, which is based on the assumption that 
software code consists of atomic elements, called function points. Each
feature has a certain amount of function points ($FP_i$), with some complexity ($C_i$). 
We multiply the amount of function points to complexities and summarize the result. What we get is
called unadjusted function points ($UFP$):

\begin{equation}
UFP = \sum_{i \in [\mbox{\scriptsize features}]} FP_i \times C_i
    = <?=$this->estimate->ufp?>
\end{equation}

Roughly, in object-oriented programming function points represent
class methods. Using these numbers we exploit
\href{http://en.wikipedia.org/wiki/COCOMO}{COCOMO-II method} 
to calculate a forecasted amount of thousands of software lines of code ($KSLoC$)
and the size of the product in staff-hours. To get the value of $KSLoC$ in
the product, we multiply the amount of function points to the
\href{http://csse.usc.edu/csse/research/COCOMOII/cocomo2000.0/CII_modelman2000.0.pdf}{$UFP$-to-$KSLoC$ convertion ratio},
which is <?=sprintf('%0.2f', $this->estimate->ufpToKsloc)?> in most our projects:

\begin{equation}
KSLoC = UFP \times <?=sprintf('%0.2f', $this->estimate->ufpToKsloc)?>
    = <?=$this->estimate->ufp?> \times <?=sprintf('%0.2f', $this->estimate->ufpToKsloc)?>
    = <?=$this->estimate->ksloc?>
\end{equation}

Here we calculate people-months ($PM$) with a formula from 
\href{http://en.wikipedia.org/wiki/COCOMO}{COCOMO-II}:

\begin{equation}\begin{gathered}
PM = A \times KSLoC^{\displaystyle(B + 0.01 \times \sum_{j=1}^5 SF_j)}
    \times {\displaystyle \prod_{i=1}^n EM_i} \\
A \times \prod_{i=1}^n EM_i = <?=sprintf('%0.2f', $this->estimate->aFactor)?> \\
B + 0.01 \times \sum_{j=1}^5 SF_j = <?=sprintf('%0.2f', $this->estimate->bFactor)?> \\
PM = <?=sprintf('%0.2f', $this->estimate->aFactor)?>
    \times <?=$this->estimate->ksloc?>^{<?=sprintf('%0.2f', $this->estimate->bFactor)?>} 
    = <?=sprintf('%0.2f', $this->estimate->pm)?>
\end{gathered}\end{equation}

Factors in the formula ($A$ and $B$) are found experimentally
after manipulating with Effort Multipliers ($EM_i$) and Scale Factors ($SF_i$). Since
our team is working with projects of similar size range and duration, the factors
are not changed very often. The result
(we multiplied $PM$ to <?=$this->estimate->daysInMonth?> --- the amount of staff-hours
in one people-month) is <?=$this->estimate->hours?> staff-hours.

