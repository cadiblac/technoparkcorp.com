%%
%% Copyright TechnoPark Corp., 2010
%% @version $Id$
%%

\section*{Iterative and Incremental Project Model}

Project manager will develop a project schedule as a collection 
of iterations and milestones. Milestones are the dates when
the team delivers some results and the project gets the next
payment. Schedule as a document is signed by you and our director.

\subsection*{Iterations}

Iterations ($I_1, I_2, \dots, I_n$) breakdown a long project onto small and manageable sub-projects: 
6--15 working days and 100--400 staff-hours. Iteration has its own
objectives, iteration plan, activities, and results. Iterations are time-boxed
pieces of work, accomplished by the project team in order to deliver
product increments ($P_1, P_2, \dots, P_n$).

\begin{tikzpicture}[font=\small]
	\setstretch{1}

	\tikzstyle{iteration} = [line width=4mm, draw=tpcBlue]

	\def\theHeight{0.5}
	\def\theWidth{1.3}

	\draw [-angle 60] (-1.3,1) -- (11,1) node [right] {time};
	\draw [-angle 60] (-1,1) -- (-1,-4) node [below] {scope};

	\draw [ultra thin, draw=tpcGrey!50] (-1,-3.8) grid [step=0.3] (10.5,1);

	\draw [iteration] (0,0) -- node [text=white] {$I_1$} (1*\theWidth,0);
	\draw [iteration] (1*\theWidth,-1*\theHeight) -- node [text=white] {$I_2$} (2*\theWidth,-1*\theHeight);
	\draw [iteration] (2*\theWidth,-2*\theHeight) -- node [text=white] {$I_3$} (3*\theWidth,-2*\theHeight);
	\draw [iteration] (3*\theWidth,-3*\theHeight) -- node [text=white] {$I_4$} (4*\theWidth,-3*\theHeight);
	\draw [iteration] (5*\theWidth,-5*\theHeight) -- node [text=white] {$I_{n-1}$} (6*\theWidth,-5*\theHeight);
	\draw [iteration] (6*\theWidth,-6*\theHeight) -- node [text=white] {$I_n$} (7*\theWidth,-6*\theHeight);

	\draw (-1.2,-1*\theHeight) node [left] {$P_2$} -- (-0.8,-1*\theHeight);
	\draw (-1.2,-2*\theHeight) node [left] {$P_3$} -- (-0.8,-2*\theHeight);
	\draw (-1.2,-3*\theHeight) node [left] {$P_4$} -- (-0.8,-3*\theHeight);
	\draw (-1.2,-5*\theHeight) node [left] {$P_{n-1}$} -- (-0.8,-5*\theHeight);
	\draw [draw=tpcGreen, fill=tpcGreen, -triangle 60] (-1.2,-6*\theHeight) node [left, draw=tpcGreen, fill=tpcGreen!20] {$P_n$} -- (-0.8,-6*\theHeight);

	\draw [triangle 60-] (0,0.8) -- (0,1.2) node [above, font=\scriptsize, text width=2cm, text centered] {Project Start};
	\draw [triangle 60-] (6.4*\theWidth,0.8) -- (6.4*\theWidth,1.2) node [above, font=\scriptsize, text width=2cm, text centered] {Project Release Milestone};
\end{tikzpicture}

\subsection*{Increments}

\tikzstyle{num} = [draw=tpcGreen, fill=tpcGreen!20, inner sep=0.5mm, outer sep=-1mm]

Product scope is elaborated, refined and delivered to the customer in 
a number of consecutive increments. Every next increment $P_i$ is a workable
product which has more implemented functions than the increment $P_{i-1}$ had.
Increment $P_n$ is a final release of the product, which is entirely accepted by
the customer.

\begin{tikzpicture}[font=\scriptsize]\setstretch{1}
	\tikzstyle{artifact} = [draw=tpcBlue, fill=tpcBlue!10, text width=2cm, text centered]
	\tikzstyle{ln} = [draw=tpcBlack!50, -angle 60]

	\def\theSize{4}

	\path [] 
		(0,0) 
		.. controls (0,1.5*\theSize) and (2*\theSize,1.5*\theSize) .. (2*\theSize,0)

		node [artifact, pos=0] (objectives) {Objectives}
		node [artifact, pos=0.1] (cost) {Cost, Schedule, WBS}
		node [artifact, pos=0.3] (risks) {Risk List (w/Risk Response Plans)}
		node [artifact, pos=0.55] (iteration) {Iteration Plan (Activities, Network Diagram, Gantt Chart)}
		node [artifact, pos=0.8] (srs) {Requirements (SRS)}
		node [artifact, pos=0.9] (sad) {Analysis Model}
		node [artifact, pos=1] (design) {Design Model};

	\path [] 
		(2*\theSize,0) 
		.. controls (2*\theSize,-1.5*\theSize) and (-\theSize,-1.5*\theSize) .. (-\theSize,0)

		node [artifact, pos=0.13] (matrix) {Traceability Matrix}
		node [artifact, pos=0.25] (code) {Code}
		node [artifact, pos=0.4] (deploy) {Deployed Product}
		node [artifact, pos=0.6] (defects) {Test Results (Defects)}
		node [artifact, pos=0.8] (changes) {Change Requests and changes to SRS}
		node [artifact, pos=1] (attributes) {Requirements Attributes};

	\node [above of=attributes] (end) {Iteration End};
	\node [below of=objectives] (start) {Iteration Start};

	\draw [ln] (start) -- (objectives);
	\draw [ln] (objectives) -- node [num] {1} (cost);
	\draw [ln] (cost) -- node [num] {2} (risks);
	\draw [ln] (risks) -- node [num] {3} (iteration);
	\draw [ln] (iteration) -- node [num] {4} (srs);
	\draw [ln] (srs) -- node [num] {5} (sad);
	\draw [ln] (sad) -- node [num] {6} (design);
	\draw [ln] (design) -- node [num] {7} (matrix);
	\draw [ln] (matrix) -- node [num] {8} (code);
	\draw [ln] (code) -- node [num] {9} (deploy);
	\draw [ln] (deploy) -- node [num] {A} (defects);
	\draw [ln] (defects) -- node [num] {C} (changes);
	\draw [ln] (changes) -- node [num] {D} (attributes);
	\draw [ln] (attributes) -- (end);

	\draw [ln] (defects) to [curve to]  node [num] {B} (code);
\end{tikzpicture}

\tikz{\node[num]{1};} The iteration starts with review of project objectives together with
constraints (cost, schedule and scope). 
\tikz{\node[num]{2};} Then we do risk identification, analysis
and response planning. 
\tikz{\node[num]{3};} According to the information from risk mitigation plans and
WBS we create a list of activities for the iteration.

\tikz{\node[num]{4};} System analysts perform the analysis of requirements,
find inconsistencies and validate the SRS for compliance with customer
expectations.

\tikz{\node[num]{5};} Software architect make changes to the Analysis Model,
adds details to UML diagrams and explain them to designers and programmers.
\tikz{\node[num]{6};} Designers detalize the Analysis Model producing
Design Model with descriptions of each class, component and module.
\tikz{\node[num]{7};} Using Traceability Matrix we update bi-directional
traceability links between requirements, UML diagrams, classes and test cases.

\tikz{\node[num]{8};} Programmers do their work, implementing the concepts
from the Analysis Model strictly according the Design Model details.
\tikz{\node[num]{9};} Programmers commit their results to Subversion repository
and continuous integration mechanism automatically deploy the product to the
deployment platform.
\tikz{\node[num]{A};} Testers finds defects in the new increment and report
them through the Defects List to the programmers.
\tikz{\node[num]{B};} Reported defects are fixed by the programmers.

\tikz{\node[num]{C};} When the Iteration is close to its finish we document 
all known changes to the requirements and document all non-fixed defects.
\tikz{\node[num]{D};} We perform User Acceptance Testing and deliver the
new product increment to the customer. Accepted requirements change their
attributes.

\subsection*{Project Team}

Project team in our typical project (e.g., 2000 staff-hours) consists of about two
dozens of full-time and part-time individuals. The most important project roles are:

\newcommand{\theBar}[2]{\tikz{\node[minimum width=#1, minimum height=3mm, draw=tpcBlue, fill=tpcBlue, anchor=east] {} node [right] {#2};}}
\begin{tabular}{lrl}
Project Role			&amp; Staff-hours &amp; \\
\hline
Programmers (3 ppl)		&amp; 400 &amp; \theBar{40mm}{20\%} \\
Tester (3 ppl)			&amp; 400 &amp; \theBar{40mm}{20\%} \\
Project Manager			&amp; 260 &amp; \theBar{26mm}{13\%} \\
Technical Reviewer (3 ppl)	&amp; 200 &amp; \theBar{20mm}{10\%} \\
Designer			&amp; 180 &amp; \theBar{18mm}{9\%} \\
Deployment Manager		&amp; 160 &amp; \theBar{16mm}{8\%} \\
Architect			&amp; 140 &amp; \theBar{14mm}{7\%} \\
Integrator			&amp; 140 &amp; \theBar{14mm}{7\%} \\
Management Reviewer		&amp; 60 &amp; \theBar{6mm}{3\%} \\
Test Analyst			&amp; 40 &amp; \theBar{4mm}{2\%} \\
Quality Reviewer		&amp; 20 &amp; \theBar{2mm}{1\%} \\
\end{tabular}

Key project participants are project manager, architect and integrator. Project manager
is responsible for planning, managing, controlling and delivering the whole project
to you. Project manager must be \href{http://www.pmi.org/CareerDevelopment/Pages/AboutPMIsCredentials.aspx}{PMP} 
certified and have at least 10,000 staff-hours of experience in our company.

Architect must be \href{http://www.sun.com/training/certification/java/scea.xml}{SCEA} or 
\href{http://www.microsoft.com/learning/mcp/architect/default.mspx}{MCA} certified, and must
prove the experience in the project subject area. Architect is responsible for the
analysis model (in UML) of the whole system and its correctness.

Integrator is responsible for source code integration into a complete workable build,
by means of builders and continuous integration tools. Integrator is a sort of chief programmer (or team leader)
in the project team. Integrator must be Sun, Microsoft or IBM certified for the key technology
used in the project (Java, .Net, C++, PHP, etc.).
