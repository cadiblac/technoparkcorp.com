%%
%% Copyright TechnoPark Corp., 2010
%% @version $Id$
%%

\section*{Rough-Order-of-Magnitude (ROM) Estimate}

We use a number of industry-wide estimating methods,
and solicit numbers from different estimators. Each method
produces its own estimates. The arithmetic average of those estimates 
is the final result. On the next <?=$this->sheet->total?>{} pages
you will find information received from estimators, with our
comments and calculations.

The bottom line is:

\begin{tabular}{>{\raggedright}p{25em}r}
    Estimator & Hours \\
    \hline
    <? foreach ($this->sheet->estimates as $estimate): ?>
        \textbf{<?=$this->tex($estimate->estimator)?>}{}
        <?=$this->tex($estimate->promo)?>{} &
        <?=$estimate->hours?>{} \\
    <? endforeach; ?>
    \hline
\end{tabular}

Math average of the numbers received is \textbf{<?=$this->sheet->hours?> staff-hours}.
Rough order of magnitude estimate of project size is
<?=$this->sheet->lowBoundary?>--<?=$this->sheet->highBoundary?>{}
staff-hours.

Read more information about <?=$this->href('process/cost/rom', 'ROM Estimate')?>{}
and our <?=$this->href('process/cost', 'two-step quotation process')?>.

Project duration preliminary estimate ($TDEV$, in calendar months)
is based on the formula from \href{http://en.wikipedia.org/wiki/COCOMO}{COCOMO},
and equals to <?=$this->sheet->tdev?> months:

\begin{equation}
TDEV = <?=$this->sheet->durationMultiplier?> \times
    \left(\frac{<?=$this->sheet->hours?>}{172}\right)^{<?=$this->sheet->durationPower?>} 
    = <?=$this->sheet->tdev?>
\end{equation}

<? foreach ($this->sheet->estimates as $estimate): ?>
    \clearpage
    \subsection*{Estimate by ``<?=$this->tex($estimate->estimator)?>''}
    
    {\setstretch{1.2}\begin{verbatim}
<?=wordwrap(implode("\n", $estimate->lines), 70)?>
    \end{verbatim}}
    
    <? $this->assign('estimate', $estimate) ?>
    <?=$this->render('ROM/' . $estimate->proposalFile)?>
<? endforeach; ?>
